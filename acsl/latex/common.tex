%\usepackage{hevea}

\usepackage[
type={CC},
modifier={by},
version={4.0}]
{doclicense}

%%% Environnements dont le corps est suprimé, et
%%% commandes dont la définition est vide,
%%% lorsque PrintRemarks=false

\usepackage{comment}
\usepackage{etoolbox}
\usepackage{xstring}

% true = prints marks signaling the state of the implementation
% false = prints only the ACSL definition, without remarks on implementation.
\newtoggle{PrintImplementationRq}
\IfEndWith*{\jobname}{implementation}
  {\toggletrue{PrintImplementationRq}}
  {\togglefalse{PrintImplementationRq}}

% true = remarks about the current state of implementation in Frama-C
% are in color.
% false = they are rendered with an underline
\newtoggle{ColorImplementationRq}
\toggletrue{ColorImplementationRq}

\newtoggle{isCPP}
\IfBeginWith*{\jobname}{acslpp}{\toggletrue{isCPP}}{\togglefalse{isCPP}}

\newcommand{\ifCPP}[1]{\iftoggle{isCPP}{\color{darkgreen}#1\color{black}}{}}
\newcommand{\ifCPPinput}[1]{\ifCPP{\input{#1}}}
%% ifCPP without color, as color does not work within grammar files
\newcommand{\ifCPPnc}[1]{\iftoggle{isCPP}{#1}{}}
\newcommand{\noCPP}[1]{\iftoggle{isCPP}{}{#1}}
\iftoggle{isCPP}
{\newcommand{\NAME}{ACSL++\xspace}\newcommand{\lang}{C++\xspace}}
{\newcommand{\NAME}{ACSL\xspace}\newcommand{\lang}{C\xspace}}

%%% Commandes et environnements pour la version relative à l'implementation
\newcommand{\highlightnotimplemented}{%
\iftoggle{ColorImplementationRq}{\color{red}}{\ul}%
}%

\newcommand{\ifImpl}[1]{\iftoggle{PrintImplementationRq}{\textit{#1}}{}}

\newcommand{\notimplemented}[2][]{%
  \iftoggle{PrintImplementationRq}{%
    \begin{changebar}%
      {\highlightnotimplemented #2}%
      \notblank{#1}{\footnote{#1}}{}%
    \end{changebar}%
  }%
  {#2}%
}

\newenvironment{notimplementedenv}[1][]{%
  \iftoggle{PrintImplementationRq}{%
    \begin{changebar}%
      \notblank{#1}{\def\myrq{#1}{}}%
      \highlightnotimplemented%
      \bgroup
    }{}
}%
{
  \iftoggle{PrintImplementationRq}{%
    \egroup%
    \ifdef{\myrq}{\footnote{\myrq}}{}%
  \end{changebar}}{}
}

%%% Environnements et commandes non conditionnelles
\newcommand{\experimental}{\textsc{Experimental}}


\newsavebox{\fmbox}
\newenvironment{cadre}
     {\begin{lrbox}{\fmbox}\begin{minipage}{0.98\textwidth}}
     {\end{minipage}\end{lrbox}\fbox{\usebox{\fmbox}}}


\newcommand{\keyword}[1]{\lstinline|#1|\xspace}
\newcommand{\keywordbs}[1]{\lstinline|\\#1|\xspace}

\newcommand{\integer}{\keyword{integer}}
\newcommand{\real}{\keyword{real}}
\newcommand{\bool}{\keyword{boolean}}

\newcommand{\Loop}{\keyword{loop}}
\newcommand{\assert}{\keyword{assert}}
\newcommand{\terminates}{\keyword{terminates}}
\newcommand{\assume}{\keyword{assume}}
\newcommand{\requires}{\keyword{requires}}
\newcommand{\ensures}{\keyword{ensures}}
\newcommand{\exits}{\keyword{exits}}
\newcommand{\returns}{\keyword{returns}}
\newcommand{\breaks}{\keyword{breaks}}
\newcommand{\continues}{\keyword{continues}}
\newcommand{\assumes}{\keyword{assumes}}
\newcommand{\frees}{\keyword{frees}}
\newcommand{\allocates}{\keyword{allocates}}
\newcommand{\invariant}{\keyword{invariant}}
\newcommand{\variant}{\keyword{variant}}
\newcommand{\assigns}{\keyword{assigns}}
\newcommand{\reads}{\keyword{reads}}
\newcommand{\decreases}{\keyword{decreases}}

\newcommand{\boundseparated}{\keywordbs{bound\_separated}}
\newcommand{\Exists}{\keywordbs{exists}~}
\newcommand{\Forall}{\keywordbs{forall}~}
\newcommand{\bslambda}{\keywordbs{lambda}~}
\newcommand{\fullseparated}{\keywordbs{full\_separated}}
\newcommand{\distinct}{\keywordbs{distinct}}
\newcommand{\Max}{\keyword{max}}
\newcommand{\nothing}{\keywordbs{nothing}}
\newcommand{\numof}{\keyword{num\_of}}
\newcommand{\offsetmin}{\keywordbs{offset\_min}}
\newcommand{\offsetmax}{\keywordbs{offset\_max}}
\newcommand{\old}{\keywordbs{old}}
\newcommand{\at}{\keywordbs{at}}

\newcommand{\If}{\keyword{if}~}
\newcommand{\Then}{~\keyword{then}~}
\newcommand{\Else}{~\keyword{else}~}
\newcommand{\For}{\keyword{for}~}
\newcommand{\While}{~\keyword{while}~}
\newcommand{\Do}{~\keyword{do}~}
\newcommand{\Let}{\keywordbs{let}~}
\newcommand{\Break}{\keyword{break}}
\newcommand{\Return}{\keyword{return}}
\newcommand{\Continue}{\keyword{continue}}

\newcommand{\exit}{\keyword{exit}}
\newcommand{\main}{\keyword{main}}
\newcommand{\void}{\keyword{void}}

\newcommand{\volatile}{\keyword{volatile}}
\newcommand{\struct}{\keyword{struct}}
\newcommand{\union}{\keywordbs{union}}
\newcommand{\inter}{\keywordbs{inter}}
\newcommand{\typedef}{\keyword{typedef}}
\newcommand{\result}{\keywordbs{result}}
\newcommand{\exception}{\keywordbs{exception}}
\newcommand{\separated}{\keywordbs{separated}}
\newcommand{\sizeof}{\keyword{sizeof}}
\newcommand{\strlen}{\keywordbs{strlen}}
\newcommand{\Sum}{\keyword{sum}}
\newcommand{\valid}{\keywordbs{valid}}
\newcommand{\validread}{\keywordbs{valid\_read}}
\newcommand{\offset}{\keywordbs{offset}}
\newcommand{\blocklength}{\keywordbs{block\_length}}
\newcommand{\baseaddr}{\keywordbs{base\_addr}}
\newcommand{\allocation}{\keywordbs{allocation}}
\newcommand{\allocable}{\keywordbs{allocable}}
\newcommand{\freeable}{\keywordbs{freeable}}
\newcommand{\fresh}{\keywordbs{fresh}}
\newcommand{\comparable}{\keywordbs{comparable}}

\newcommand{\N}{\ensuremath{\mathbb{N}}}
\newcommand{\ra}{\ensuremath{\rightarrow}}
\newcommand{\la}{\ensuremath{\leftarrow}}


% BNF grammar
\newcommand{\indextt}[1]{\index{#1@\protect\keyword{#1}}}
\newcommand{\indexttbs}[1]{\index{#1@\protect\keywordbs{#1}}}
\newif\ifspace
\newif\ifnewentry
\newcommand{\addspace}{\ifspace ~ \spacefalse \fi}
\newcommand{\term}[2]{\addspace\hbox{\lstinline|#1|%
\notblank{#2}{\indexttbase{#2}{#1}}{}}\spacetrue}
\newcommand{\nonterm}[2]{%
  \ifblank{#2}%
   {\addspace\hbox{\textsl{#1}\ifnewentry\index{grammar entries!\textsl{#1}}\fi}\spacetrue}%
   {\addspace\hbox{\textsl{#1}\footnote{#2}\ifnewentry\index{grammar entries!\textsl{#1}}\fi}\spacetrue}}
\newcommand{\repetstar}{$^*$\spacetrue}
\newcommand{\repetplus}{$^+$\spacetrue}
\newcommand{\repetone}{$^?$\spacetrue}
\newcommand{\lparen}{\addspace(}
\newcommand{\rparen}{)}
\newcommand{\orelse}{\addspace$\mid$\spacetrue}
\newcommand{\sep}{ \\[2mm] \spacefalse\newentrytrue}
\newcommand{\newl}{ \\ & & \spacefalse}
\newcommand{\alt}{ \\ & $\mid$ & \spacefalse}
\newcommand{\is}{ & $::=$ & \newentryfalse}
\newenvironment{syntax}{\begin{tabular}{@{}rrll@{}}\spacefalse\newentrytrue}{\end{tabular}}
\newcommand{\synt}[1]{$\spacefalse#1$}
\newcommand{\emptystring}{$\epsilon$}
\newcommand{\below}{See\; below}

% colors

\definecolor{darkgreen}{rgb}{0, 0.5, 0}

% theorems

\newtheorem{example}{Example}[chapter]

%%% Local Variables:
%%% mode: latex
%%% TeX-PDF-mode: t
%%% TeX-master: "main"
%%% End:


%Already set in frama-c-book
%\usepackage[a4paper=true,pdftex,colorlinks=true,urlcolor=blue,pdfstartview=FitH]{hyperref}

%already set in frama-c-book
%\usepackage[T1]{fontenc}
%\usepackage{times}
\usepackage{amssymb}
\usepackage{graphicx}
\usepackage{enumitem}
\newcommand{\itembf}{\item[\stepcounter{enumi}\textbf{\arabic{enumi}.}]}
%\usepackage{tikz}
\usepackage{color}
\usepackage{xspace}
\usepackage{makeidx}
\usepackage[normalem]{ulem}
\usepackage[leftbars]{changebar}
\usepackage{alltt}
\usepackage{hyperref}
\usepackage{multirow}
\usepackage{fixme}
% For reports, use disable to remove all todonotes
\usepackage[textsize=footnotesize, bordercolor=orange, backgroundcolor=none]{todonotes}

\lstset{
numbers=left,
deletekeywords={assert}
}
\makeindex

%Already set in frama-c-book
%\setlength{\textheight}{240mm}
%\setlength{\topmargin}{-10mm}
%\setlength{\textwidth}{160mm}
%\setlength{\oddsidemargin}{0mm}
%\setlength{\evensidemargin}{0mm}

\renewcommand{\textfraction}{0.01}
\renewcommand{\topfraction}{0.99}
\renewcommand{\bottomfraction}{0.99}

\newcommand{\fclang}{\textsc{Frama-Clang}\xspace}
\newcommand{\framac}{\textsc{Frama-C}\xspace}
\newcommand{\acsl}{\textsc{ACSL}\xspace}
\newcommand{\acslb}{\acsl/\NAME\xspace}

\newcommand{\TODO}[1]{\fxnote[margin]{TODO: #1}}
\newcommand{\review}[1]
{\fxnote[inline,nomargin]{\color{darkgreen}Question for Reviewers: #1}}

% already taken care of in frama-c-book
% \usepackage{fancyhdr}
% \pagestyle{fancyplain}
% \renewcommand{\footrulewidth}{0.4pt}
% \addtolength{\headheight}{2pt}
% \addtolength{\headwidth}{1cm}
% \renewcommand{\chaptermark}[1]{\markboth{#1}{}}
% \renewcommand{\sectionmark}[1]{\markright{\thesection\ #1}}
% \lhead[\fancyplain{}{\bfseries\thepage}]{\fancyplain{}{\bfseries\rightmark}}
% \chead{}
% \rhead[\fancyplain{}{\bfseries\leftmark}]{\fancyplain{}{\bfseries\thepage}}
% \lfoot{\fancyplain{}{ANSI/ISO C Specification Language}}
% \cfoot{}
% \rfoot{\fancyplain{}{CAT RNTL project}}

% Floating-point commands and notations
\usepackage{amsmath}
\DeclareMathOperator{\ulp}{ulp}
\DeclareMathOperator{\emin}{emin}
